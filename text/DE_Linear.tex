
\section{First Order Linear Differential Equations}\label{sec:Linear}

In the previous section, we explored a specific techique to solve a specific type of differential equation; a separable differential equation.  In this section, we develop and practice a technique to solve a type of differential equation called a \emph{first order linear} differential equation.

Recall than an algebraic equation in one variable is one that can be written $ax + b = 0,$ where $a$ and $b$ are real numbers.  Notice that the variable $x$ appears to the first power.  The equations $\sqrt{x}+1=0$ and $\sin(x)-3x = 0$ are both nonlinear. A linear differential equation is one in which the dependent variable and its derivatives appear only to the first power.  We focus on first order equations, which first (but not higher order) derivatives of the independent variable.

\definition{def:Linear}{First Order Linear Differential Equation}
{A \textbf{first order linear differential equation} is a differential equation that can be written in the form
	\[
	\frac{dy}{dx} + p(x)y = q(x),
	\]
where $p$ and $q$ are arbitrary functions of the independent variable $x$.
}

\example{ex_identify_linear}{Classifying Differential Equations}{
Classify the followind differential equations as first order linear, separable, both, or neither.\\
\begin{minipage}[h]{.5\linewidth}
	\begin{enumerate}
		\item[(a)] $\displaystyle \frac{dy}{dx} = xy$
		\item[(b)] $\displaystyle \frac{dy}{dx} = e^y + 3x$
	\end{enumerate}	
\end{minipage}%
\begin{minipage}[h]{.5\linewidth}
	\begin{enumerate}
		\item[(c)] $\displaystyle \frac{dy}{dx} - (\cos x)y = \cos x$
		\item[(d)] $\displaystyle y\frac{dy}{dx} -3xy = 4\ln x$
	\end{enumerate}	
\end{minipage}
}
{(a)\ Both.  We identify $p(x) = -x$ and $q(x) = 0.$  The separated form of the equation is $\displaystyle \frac{dy}{y} = x\,dx.$\\
(b)\ Neither.  The $e^y$ term makes the equation nonlinear.  Because of the addition, it is not possible to write the equation in separated form.\\
(c)\ First order linear.  We identify $p(x) = -\cos x$ and $q(x) = \sin x.$  The equation cannot be written in separated form.\\
(d)\ Neither.  Notice that dividing by $y$ results in the nonlinear term $\displaystyle \frac{4\ln x}{y}.$ It is not possible to write the equation in separated form.
}\\

Notice that linearity depends on the dependent variable $y$, not the independent variable $x$.  The functions $p(x)$ and $q(x)$ need not be linear as demonstrated in part (c) of example \ref{ex_identify_linear}.  Neither $\cos x$ nor $\sin x$ are linear functions of $x$, but the differential equation is still linear.

\vskip\baselineskip
\noindent\textbf{\large Solving First Order Linear Equations}
\vskip\baselineskip

We motivate the solution technique by way of an observation and an example.  We first observe that the expression $\displaystyle \frac{d}{dx}\big(xy\big)$ can be expanded via the product rule and implicit differentiation to the expression $\displaystyle x \frac{dy}{dx} + y.$ Now we look at an example.  Consider the first order linear differential equation
	\[
		\frac{dy}{dx} + \frac{1}{x}y = \frac{\sin x \cos x}{x}.
	\]
If we multiply both sides of the differential equation by $x$ and use our observation, we see that the differential equation can be written
	\[
		\frac{d}{dx}\big(xy\big) = \sin x \cos x .
	\]
Integrating both sides of the equation with respect to $x$ and using the substitution $u = \sin x$ on the right results in the implicit solution
	\[
		xy = \frac{1}{2}\sin^2 x + C.
	\]
Solving for $y$ yields the explicit solution
	\[
		y = \frac{\sin^2 x}{2x} + \frac{C}{x}.
	\]
	
\mnote{.45}{\textbf{Note:} In the examples in the previous section, we performed operations on the arbitrary constant $C$, but still called the result $C$. The justification is that the result after the operation is \emph{still} an arbitrary contant.  Here, we divide $C$ by $x$, so the result depends explicitly on the independent variable $x$.  Since $C/x$ is \emph{not} contant, we can't just call it $C$.}

As motivated by the problem we just solved, the basic idea behing solving first order linear differential equations is to multiply both sides of the differential equation by a function, called an \emph{integrating factor}, that makes the left hand side of the equation look like an expanded product rule.  We then condense the left hand side into the derivative of a product and integrate both sides. An obvious question is how to find the integrating factor.

Consider the first order linear equation
	\[
		\frac{dy}{dx} + p(x)y = q(x).
	\]
Let's call the integrating factor $\mu(x)$.  We multiply both sides of the differential equation by $\mu(x)$ to get
	\[
		\mu(x) \left ( \frac{dy}{dx} + p(x)y \right ) = \mu(x)q(x).
	\]
Our goal is to choose $\mu(x)$ so that the left hand side of the differential equation looks like the result of a product rule.  The left hand side of the equation is
	\[
		\mu(x) \frac{dy}{dx} + \mu(x)p(x)y,
	\]
and using the product rule and implicit differentiation,
	\[
		\frac{d}{dx} \big ( \mu(x) y \big ) = \frac{d\mu}{dx}y + \mu(x)\frac{dy}{dx}.
	\]
Equating these two gives
	\[
		\frac{d\mu}{dx}y + \mu(x)\frac{dy}{dx} = \mu(x) \frac{dy}{dx} + \mu(x)p(x)y,
	\]
or
	\[
		\frac{d\mu}{dx} = \mu(x)p(x).	
	\]
In order for the integrating factor $\mu(x)$ to perform its job, it must solve the differential equation above.  But that differential equation is separable, so we can solve it.  The separated form is
	\[
		\frac{d\mu}{\mu} = p(x)\,dx.
	\]
Integrating,
	\[
		\ln \mu = \int p(x)\,dx,
	\]
or
	\[
		\mu(x) = e^{\int p(x)\,dx}.
	\]
If $\mu(x)$ is chosen this way, the differential equation can be written in the form
	\[
		\frac{d}{dx} \big( \mu(x)y \big) = \mu(x)q(x).
	\]
Integrating and solving for $y$, the explicit solution is
	\[
		y = \frac{1}{\mu(x)}\int \big( \mu(x)q(x) \big)\,dx.
	\]
	
\mnote{.45}{\textbf{Note:} Following the steps outlined in the previous section, we should technically end up with $\mu(x) = Ce^{\int p(x)\,dx},$ where $C$ is an arbitrary constant.  Because we multiply both sides of the differential equation by $\mu(x)$, the arbitrary constant cancels, and we omit it when finding the integrating factor. }

Though this formula can be used to write down the solution to a first order linear equation, the process is lost by simply memorizing a formula, and always always follow the steps outlined in key idea \ref{idea:solving_linear} when solving equations of this type.

\keyidea{idea:solving_linear}{Solving First Order Linear Equations}
{
\begin{enumerate}
	\item Write the differential equation in the form
		\[
			\frac{dy}{dx} + p(x)y = q(x).
		\]
	\item Compute the integrating factor
		\[
			\mu(x) = e^{\int p(x)\,dx}.
		\]
	\item Multiply both sides of the differential equation by $\mu(x)$, and condense the left hand side to get
		\[
			\frac{d}{dx}\big( \mu(x)y \big) = \mu(x)q(x).
		\]
	\item Integrate both sides of the differential equation with respect to $x$, taking care to remember the arbitrary constant.
	\item Solve for $y$ to find the explicit solution to the differential equation.
\end{enumerate}
}

Let's practice the process by solving the two first order linear differential equations from example \ref{ex_identify_linear}.\\

\example{ex_linear1}{Solving a First Order Linear Equation}{
Find the general solution to $\displaystyle \frac{dy}{dx} = xy.$}
{We solve by following the steps in key idea \ref{idea:solving_linear}. Unlike the process for solving separable equations, we need not worry about losing constant solutions.  The answer we find \emph{will} be the general solution to the differential equation. We first write the equation in the form
	\[
		\frac{dy}{dx} - xy = 0.
	\]
By identifying $p(x) = -x,$ we can compute the integrating factor
	\[
		\mu(x) = e^{\int -x\,dx} = e^{-\frac{1}{2}x^2}.
	\]
Multiplying both side of the differential equation by $\mu(x),$ we have
	\[
		e^{-\frac{1}{2}x^2}\left( \frac{dy}{dx} - xy\right) = 0.
	\]
The left hand side of the differential equation condenses to yield
	\[
		\frac{d}{dx} \left ( e^{-\frac{1}{2}x^2}y\right ) = 0.
	\]
We integrate both sides with respect to $x$ to find
	\[
		e^{-\frac{1}{2}x^2}y = C,
	\]
or the explicit solution
	\[
		y  = Ce^{-\frac{1}{2}x^2}.
	\]
}

\mnote{.825}{\textbf{Note:} The step where the left hand side of the differential equation condenses to the derivative of a product can feel a bit magical.  The reality is that we choose $\mu(x)$ so that we can get exactly this condensing behavior.  It's not magic, it's math!  If you're still skeptical, try using the product rule and implicit differentiation to evaluate $\displaystyle \frac{d}{dx}\left (e^{-\frac{1}{2}x^2}y\right)$, and verify that it becomes $e^{-\frac{1}{2}x^2}\left(\displaystyle \frac{dy}{dx} - xy\right)$.}

\vskip\baselineskip

\example{ex_linear2}{Solving a First Order Linear Equation}{
Find the general solution to $\displaystyle \frac{dy}{dx} -(\cos x)y = \cos x$.}
{The differential equation is already in the correct form.  The integrating factor is given by
	\[
		\mu(x) = e^{-\int\cos x\,dx} = e^{-\sin x}.
	\]
Multiplying both sides of the equation by the integrating factor and condensing,
	\[
		\frac{d}{dx}\left(e^{-\sin x}y \right) = (\cos x) e^{-\sin x}
	\]
Using the substitution $u = -\sin x$, we can integrate to find the implicit solution
	\[
		e^{-\sin x} y = -e^{-\sin x} + C.
	\]
The explicit form of the general solution is
	\[
		y = -1 + Ce^{\sin x}.
	\]
}

%\printexercises{exercises/DE_03_exercises}