One of the strengths of calculus is its ability to describe real-world phenomena.  We have seen hints of this in our discussion of the applications of derivatives and integrals in the previous chapters.  The process of formulating an equation or multiple equations to describe a physical phenomenon is called \emph{mathematical modeling.} As a simple example, populations of bacteria are often described as ``growing exponentially." Looking in a biology text, we might see $P(t) = P_0e^{kt},$ where $P(t)$ is the bacteria population at time $t$, $P_0$ is the initial population at time $t=0$, and the constant $k$ describes how quickly the population grows. This equation for exponential growth arises from the assumption that the population of bacteria grows at a rate proportional to its size.  Recalling that the derivative gives the rate of change of a function, we can describe the growth assumption precisely using the equation $P' = kP.$ This equation is called a \emph{differential equation}, and is the subject of the current chapter.

\vskip-\baselineskip
\section{Graphical and Numerical Solutions to Differential Equations}\label{sec:Graphical_Numerical}
In section \ref{sec:antider}, we were introduced to the idea of a differential equation.  Given a function $y = f(x),$ we defined a \emph{differential equation} as an equation involving $y, x,$ and derivatives of $y$. We explored the simple differential equation $y' = 2x,$
and saw that a \emph{solution} to a differential equation is simply a function that satisfies the differential equation.

\vskip\baselineskip
\noindent\textbf{\large Introduction and Terminology}
\vskip\baselineskip

\definition{def:ODE}{Differential Equation}
{Given a function $y=f(x)$, a \textbf{differential equation} is an equation relating $x, y,$ and derivatives of $y$.
\begin{itemize}
\item The variable $x$ is called the \textbf{independent variable}.
\item The variable $y$ is called the \textbf{dependent variable}.
\item The \textbf{order} of the differential equation is the order of the highest derivative of $y$.
\end{itemize}
}

Let us return to the simple differential equation
\[
y' = 2x.
\]
To find a solution, we must find a function whose derivative is $2x$.  In other words, we seek an antiderivative of $2x.$  The function 
\[y = x^2\]
is an antiderivative of $2x$, and solves the differential equation.  So do the functions
\[y = x^2 + 1\]
and
\[y = x^2 - 2346.\]
We call the function
\[y = x^2 +  C,\]
with $C$ an arbitrary constant of integration, the \emph{general solution} to the differential equation.

In order to specify the value of the integration constant $C$, we require additional information.  For example, if we know that $y(1) = 3$, it follows that $C=2$.  This additional information is called an \emph{initial condition.}

\definition{def:IVP}{Initial Value Problem}
{A differential equation paired with an initial condition (or initial conditions) is called an \textbf{initial value problem.}\\

The solution to an initial value problem is called a \textbf{particular solution} to the initial value problem.

The solution to a differential that encompasses all possible solutions is called the \textbf{general solution} to the differential equation.
}\\

\example{ex_simple_de}{A simple first-order differential equation}{
Solve the differential equation $\displaystyle y' = 2y$.}
{The solution is a function $y$ such that differentiation yields twice the original function.  Unlike our starting example, finding the solution here does not involve computing an antiderivative.  Notice that ``integrating both sides'' would yield the result $y = \int 2y\,dx,$ which is not useful.  Without knowledge of the function $y$, we can't compute the indefinite integral. Later sections will explore systematic ways to find analytic solution to simple differential equations.  For now, a bit of though might let us guess that solution 
\[y = e^{2x}.\]
Notice that application of the chain rule yields $y' = 2e^{2x} = 2y.$ Another solution is given by
\[y = -3e^{2x}.\]
In fact
\[y = Ce^{2x},\]
where $C$ is any constant, is the \emph{general solution} to the differential equation because $y' = 2Ce^{2x} = 2y$.

If we are provided with a single initial condition, say $y(0) = 3,$ we can identify $C=3$ so that
\[y = 3e^{2x}\]
is the \emph{particular solution} to the initial value problem 
\[y' = 2y, \text{ with } y(0) = 3.\]
\vskip -\baselineskip
}\\

\example{ex_simple_de2}{A second-order differential equation}{
Solve the differential equation $y'' + 9y = 0.$}
{We seek a function such that two derivatives returns negative 9 multiplied by the original function.  Both $\sin(3x)$ and $\cos(3x)$ have this feature.  The general solution to the differential equation is given by
\[y = C_1\sin(3x) + C_2\cos(3x),\]
where $C_1$ and $C_2$ are arbitrary constants.  To fully specify a particular solution, we require two additional conditions.  For example, the initial conditions $y(0)=1$ and $y'(0)=3$ yield $C_1 = C_2 = 1.$}\\

The differential equation in example \ref{ex_simple_de2} is second order because the equation involves a second derivative.  In general, the number of initial conditions required to specify a particular solution depends on the order of the differential equation.  For the remainder of the chapter, we restrict our attention to first order differential equations and first order initial value problems.

\vskip\baselineskip
\example{ex_verify_solutions}{Verifying a solution to the differential equation}{
Which of the following is a solution to the differential equation \[ y' + \frac{y}{x} - \sqrt{y} = 0?\]
\begin{center}
\hfill a) $y = C \left ( 1 + \ln x \right )^2$ \hfill b) $y = \left ( \frac{1}{3}x + \frac{C}{\sqrt{x}} \right )^2$ \hfill c) $y = C e^{-3x} + \sqrt{\sin x}$ \hfill
\end{center}}
{Verifying a solution to a differential equation is simply an exercise in differentiation and simplification.  We substitute equation solution in the differential equation to see if it satisfies the equation.

a) Testing the potential solution $y = C \left ( 1 + \ln x \right )^2$:

Differentiating, we have $\displaystyle y' = \frac{2C(1 + \ln x)}{x}$.  Substituting into the differential equation,

}\\


\vskip\baselineskip
\noindent\textbf{\large Graphical Solutions to Differential Equations}
\vskip\baselineskip

The solutions to the differential equations we have found so far are called \emph{analytic solutions.} We have found exact forms for the functions that solve the  differential equations.  Many times a differential equation will have a solution, but it is difficult or impossible to find the solution analytically.  This is analogous to algebraic equations.  The algebraic equation $x^2 + 3x - 1 = 0$ has two real solutions that can be found analytically by using the quadratic formula. The equation $\cos x = x$ has one real solution, but we can find it analytically. As shown in figure \ref{fig:DE_01_intersection}, we can find an approximate solution graphically by plotting $\cos x$ and $x$ and observing the $x$-value of the intersection. We can similarly use graphical tools to understand the qualitative behavior of solutions to a first order-differential equation.

\mfigure{.81}{Graphically finding an approximate solution to $\cos x = x.$}{fig:DE_01_intersection}{figures/figDE_01_intersection}

Consider the first-order differential equation
\[y' = f(x,y).\]
The function $f$ could be any function of the two variables $x$ and $y$.  Written in this way, we can think of the function $f$ as providing a formula to find the slope of a solution at a given point in the $xy$-plane.  In other words, suppose a solution to the differential equation passes through the point $(x_0,y_0)$.  Then, at the point $(x_0,y_0)$ the slope of the solution curve will be $f(x_0,y_0).$  Since this calculation of the slope is possible at any point $(x,y)$ where the function $f(x,y)$ is defined, we can produce a plot called a \emph{slope field} that shows the slope of a solution at any point in the $xy$-plane where the solution is defined.  Further, this process can be done purely by working with the differential equation itself.  In other words, we can draw a slope field and use it to determine the qualitative behavior of solutions to a differential equation without having to solve the differential equation.

\definition{def:slope_field}{Slope Field}
{A \textbf{slope field} for a first-order differential equation $y' = f(x,y)$ is a plot in the $xy$-plane made up of short line segments or arrows. For each point $(x_0,y_0)$ where $f(x,y)$ is defined, the slope of the line segment is given by $f(x_0,y_0)$}\\

\example{ex_slope_field}{Finding a slope field}{
Find a slope field for the differential equation $\displaystyle y' = x+y$.}
{Because the function $f(x,y) = x+y$ is defined for all points $(x,y)$, every point in the $xy$-plane has an associated line segment. It is not practical to draw an entire slope field by hand, but many tools exist for drawing slope fields on a computer.  We will explicitly calculate and plot a few of the line segments in the slope field.\\

\noindent The slope of the line segment at $(0,0)$ is given by $f(0,0) = 0 + 0 = 0.$\\

\noindent The slope of the line segment at $(1,1)$ is given by $f(1,1) = 1 + 1 = 2.$\\

\noindent The slope of the line segment at $(1,-1)$ is given by $f(1,-1) = 1 - 1 = 0.$\\

\noindent The slope of the line segment at $(-2,3)$ is given by $f(-2,3) = -2 + 3 = 1.$\\

These four components of the slope field are shown in figure .  The entire slope field for the differential equation is shown in figure .
\vskip -\baselineskip
}\\

\example{ex_IVP_graphical}{Finding a graphical solution to an initial value problem}{
Find a graphical solution to the initial value problem $\displaystyle y' = x+y$ with $y(1)=-1$.}
{The solution to the initial value problem should be a continuous smooth curve.  Using the slope field, we can draw of a sketch of the solution using the following two criteria:

\begin{enumerate}
\item The solution must pass through the point $(1,-1)$.
\item When the solution passes through a point $(x_0,y_0)$ it must be tangent to the line segment at $(x_0,y_0)$.
\end{enumerate}
Essentially, we sketch a solution to the initial value problem by starting at the point $(1,-1)$ and ``following the lines" in either direction.  A sketch of the solution is shown in figure .

\vskip -\baselineskip
}\\

