
\section{Modeling with Differential Equations}\label{sec:Modeling}

In the first three sections of this chapter, we focused on the basic ideas behind differential equations and the mechanics of solving certain types of differential equations.  We have only hinted at their practical use.  In this section, we use differential equations for mathematical modeling, the process of using equations to describe real world processes.  We explore a few different mathematical models with the goal of gaining an introduction to this large field of applied mathematics.

\vskip\baselineskip
\noindent\textbf{\large Models Involving Proportional Change}
\vskip\baselineskip

Some of the simplest differential equation models involve one quantity that changes at a rate proportional to another quantity.  In the introduction to this chapter, we consider a population that grows at a rate proportional to the current population.  The words in this assumption can be directly translated into a differential equation as shown below.

\begin{center}
	\myincludegraphics{figures/figDE_04_translation}
	\captionsetup{type=figure}%
	\caption{Translating words into a differential equation.}\label{fig:DE_translation}
\end{center}

There are some key ideas that can be helpful when translating words into a differential equation.  Any time we see something about rates or changes, we should think about derivatives.  The word ``is" usually corresponds to an equal sign in the equation.  The words ``proportional to" mean we have a constant multiplied by something.

The differential equation in figure \ref{fig:DE_translation} is easily solved using separation of variables. We find
	\[
		p = Ce^{kt}.
	\]
Notice that we need values for both $C$ and $k$ before we can use this formula to predict population size.  We require information about the population at two different times in order to fully determine the population model.

\vskip\baselineskip

\enlargethispage{\baselineskip}

\example{ex_ecoli}{Bacterial Growth}{
Suppose a population of \emph{e-coli} bacteria grows at a rate proportional to the current population.  If an initial popluation of 200 bacteria has grown to 1600 three hours later, find a function for the size of the population at time $t$, and use it to predict when the population size will reach 10,000.}
{We already know that the population at time $t$ is given by $p = Ce^{kt}$ for some $C$ and $k$.  The information about the initial size of the population means that $p(0)=200.$ Thus $C=200.$  Our knowledge of the population size after three hours allows us to solve for $k$ via the equation
	\[
		1600 = 200e^{3k}.
	\]
Solving this exponential equation yields $k =\ln(8)/3 \approx 0.6931.$  The popluation at time $t$ is given by
	\[
		p = 200 e^{(\ln(8)/3)t}.
	\]
Solving
	\[
		10000 = 200e^{(\ln(8)/3)t}
	\]
yields $t =(3\ln 50)/\ln 8 \approx 5.644.$  The population is predicted to reach 10,000 bacteria in slightly more than five and a half hours.
}\\

Another example of porportional change is \textbf{Newton's Law of Cooling.} The laws of thermodynamics state that heat flows from areas of high temperature to areas of lower temperature.  A simple example is a hot object that cools down when placed in a cool room.  Newton's Law of Cooling is the simple assumption that the temperature of the object changes at a rate proportional to the difference between the temperature of the object and the ambient temperature of the room.  If $T$ is the temperature of the object, and $A$ is the constant ambient temperature, Newton's Law of Cooling can be expressed as the differential equation
	\[
		\frac{dT}{dt} = k(A - T.)
	\]
This differential equation is both linear and separable. The separated form is
	\[
		\frac{1}{A-T}\,dT = k\,dt.
	\]
Then an implicit definition of the temperature is given by
	\[
		-\ln|A-T| = kt + C.
	\]
If we solve for $T$, we find the explicit temperature
	\[
		T = A-Ce^{-kt}.
	\]
Though we didn't show the steps, the explicit solution involves the typical process of renaming the constant $\pm e^{-C}$ as $C$, and allowing $C$ to be positive, negative, or zero to account for both cases of the absolution value and to catch the constant solution $T=A.$

\mnote{.5}{\textbf{Note:} The equation $\displaystyle \frac{dT}{dt} = k(T-A)$ is also a valid representation of Newton's Law of Cooling.  Intuition tells us that $T$ will increase if $T$ is less than $A$ and decrease if $T$ is greater then $A$.  The form we use in the text follows this intuition with a positive $k$ value.  The form above will require that $k$ take on a negative value. In the end, both forms result in the same general solution.}

\example{ex_coffee}{Hot Coffee}{A freshly brewed cup of coffee is set on the counter and has a temperature of 200$^\circ$ fahrenheit.  After 3 minutes, it has cooled to 190$^\circ$, but is still too hot to drink.  If the room is 72$^\circ$ and the coffee cools according to Newton's Law of Cooling, how long will the impatient coffee drinker have to wait until the coffee has cooled to 165$^\circ$?}
{Since we have already solved the differential equation for Newton's Law of Cooling, we can immediately use the function
	\[
		T = A - Ce^{-kt}.
	\]
Since the room is 72$^\circ$, we know $A = 72.$  The initial temperature is 200$^\circ$, which means $C = -128.$  At this point, we have
	\[
		T = 72 + 128e^{-kt}
	\]
The information about the coffee cooling to 190$^\circ$ in 3 minutes leads to the equation
	\[
		190 = 72 + 128e^{-3k}.
	\]
solving the exponential equation for $k$, we have
	\[
		k = -\frac{1}{3}\ln \left(\frac{59}{64}\right) \approx 0.0271.
	\]
Finally, we finish the problem by solving the exponential equation
	\[
		165 = 72 + 128e^{\frac{1}{3}\ln \left(\frac{59}{64}\right)t}.
	\]
The coffee drinker must wait $\displaystyle t = \frac{3 \ln \left(\frac{93}{128}\right)}{\ln \left(\frac{59}{64}\right)} \approx 11.78$ minutes.
}\\

We finish our discussion of models of proportional change by exploring three different models of disease spread through a population.  In all of the models, we let $y$ denote the proportion of the population that is sick ($0 \leq y \leq 1$).  We assume a proportion of $0.05$ is initially sick and that a proportion of $0.1$ is sick 1 week later.

\vskip\baselineskip

\example{ex_disease_exponential}{Disease Spread 1}{Suppose a disease spreads through a population at a rate proportional to the number of individuals who are sick.  If 5\% of the population is sick initially and 10\% of the population is sick one week later, find a formula for the proportion of the popoulation that is sick at time $t$.}
{The assumption here seems to have some merit because it matches our intuition that a disease should spread more rapidly when more individuals are sick.  The differential equation is simply
	\[
		\frac{dy}{dt} = ky,
	\]
with solution
	\[
		y = Ce^{kt}.
	\]
The conditions $y(0)=0.05$ and $y(1) = 0.1$ lead to $C = 0.05$ a and $k = \ln 2$, so the function is
	\[
		y = 0.05e^{(\ln 2)t}.
	\]
We should point out a glaring problem with this model.  The variable $y$ is a proportion and should take on values between 0 and 1, but the function $y = 0.05e^{2t}$ grows without bound.  After $t \approx 4.32$ weeks, $y$ exceeds 1, and the model ceases to make physical sense.
}\\

\example{ex_disease_newton}{Disease Spread 2}{Suppose a disease spreads through a population at a rate proportional to the number of individuals who are not sick.  If 5\% of the population is sick initially and 10\% of the population is sick one week later, find a formula for the proportion of the popoulation that is sick at time $t$}
{The intuition behind the assumption here is that a disease can only spread if there are individuals who are susceptible to the infection.  As fewer and fewer people are able to be infected, the disease spread should slow down.  Since $y$ is proportion of the population that is sick, $1-y$ is the proportion who are not sick, and the differential equation is
	\[
		\frac{dy}{dt} = k(1-y).
	\]
Though the context is quite different, the differential equation is identical to the differential equation for Newton's Law of Cooling, with $A=1$.  The solution is
	\[
		y = 1 - Ce^{-kt}.
	\]
The conditions $y(0)=0.05$ and $y(1) = 0.1$ yield $C = 0.95$ and $k = -\ln\left(\frac{18}{19}\right) \approx 0.0541,$ so the final function is
	\[
		y = 1-.95e^{\ln\left(\frac{18}{19}\right)t}.
	\]
Notice that this function approches $y=1$ in the limit as $t \to \infty,$ and does not suffer from the non-physical behavior described in example \ref{ex_disease_exponential}.
}\\

In example \ref{ex_disease_exponential}, we assume disease spread depends on the number of infected individuals.  In example \ref{ex_disease_newton}, we assume disease spread depends on the number of susceptible individuals who are able to become infected.  In reality, we would expect many diseases to require the interaction of both infected on susceptible individuals in order to spread.  One of the simplest ways to model this required interaction is to assume disease spread depends on the product of the proportions of infected and uninfected individuals.  This assumption is often called the \emph{law of mass action.}\\

\example{ex_disease_logistic}{Disease Spread 3}{Suppose a disease spreads through a population at a rate proportional to the product of the number of infected and uninfected individuals. If 5\% of the population is sick initially and 10\% of the population is sick one week later, find a formula for the proportion of the popoulation that is sick at time $t$}
{The differential equation is
	\[
		\frac{dy}{dt} = ky(1-y).
	\]
This is exactly the logistic equation with $M = 1.$  We solve this differential equation in example \ref{ex_logistic}, and find
	\[
		y = \frac{1}{1 + be^{-kt}}.
	\]
The conditions $y(0)=0.05$ and $y(1) = 0.1$ yield $b = 19$ and $k = -\ln\left(\frac{9}{19}\right) \approx 0.7472.$  The final functionis
	\[
		y = \frac{1}{1+19e^{\ln\left(\frac{9}{19}\right)t}}.
	\]
}

\mtable{.55}{Plots of the functions from example \ref{ex_disease_exponential} (dotted), example \ref{ex_disease_newton} (dashed), and example \ref{ex_disease_logistic} (solid).}{fig:disease}{\begin{tabular}{c} \myincludegraphics{figures/figDE_04_disease} \\ \rule{0pt}{10pt}(a) \\ \\ \myincludegraphics{figures/figDE_04_disease1}  \\  \rule{0pt}{10pt}(b) \end{tabular}}

%\printexercises{exercises/DE_03_exercises}