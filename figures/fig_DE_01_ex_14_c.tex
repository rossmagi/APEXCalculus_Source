\def\length{sqrt(1+(-y)^2)}
\begin{tikzpicture}
\begin{axis}[width=\marginparwidth+25pt,
tick label style={font=\scriptsize},
%axis y line=middle,axis x line=middle,
axis lines = center,
name=myplot,%
view={0}{90},
domain = -2:2,
y domain = -2:2,
%			xtick={-2,-1,..,2},%
%			xticklabels={-2,-1,..,2},
			ytick={-3,-2,-1,0,1,2,3},
%			yticklabels = {-2,-1,0,1,2},
%			minor x tick num=1,
%			extra x ticks={.25,.75},%
%			ymin=-.1,ymax=1.1,%
%			xmin=-.1,xmax=1.1%
]
\addplot3 [{\colortwo},quiver = {u = {1/(\length)}, v = {(-y)/(\length)},scale arrows=.1},samples=15]{0};
\end{axis}
\node [right] at (myplot.right of origin) {\scriptsize $x$};
\node [above] at (myplot.above origin) {\scriptsize $y$};
\end{tikzpicture}
