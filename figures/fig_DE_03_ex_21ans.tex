\def\length{sqrt(1+(x - y)^2)}
\begin{tikzpicture}
\begin{axis}[width=\marginparwidth+25pt,
tick label style={font=\scriptsize},
%axis y line=middle,axis x line=middle,
axis lines = center,
name=myplot,%
view={0}{90},
y domain = -.15:4,
domain = -.15:4,
%			xtick={0,1,..,10},%
%			xticklabels={-1,0,1,2,3,4,5},
%			ytick={-2,-1,0,1,2},
%			yticklabels = {-2,-1,0,1,2},
%			minor x tick num=1,
%			extra x ticks={1,3},%
%			minor y tick num=1,
%			extra y ticks={2.5,7.5},
%			ymin=-.1,ymax=1.6,%
%			xmin=-.1,xmax=5.2%
]
\addplot3 [{\colortwo},quiver = {u = {1/(\length)}, v = {(x-y)/(\length)},scale arrows=.2},samples=15]{0};
\addplot [{\colorone},thick,domain=0:4.1,samples=50] {x-1+exp(-x)};
\end{axis}
\node [right] at (myplot.right of origin) {\scriptsize $x$};
\node [above] at (myplot.above origin) {\scriptsize $y$};
\end{tikzpicture}
