\def\length{sqrt(1+(y*(1-y))^2)}
\begin{tikzpicture}
\begin{axis}[width=\marginparwidth+25pt,
tick label style={font=\scriptsize},
%axis y line=middle,axis x line=middle,
axis lines = middle,
name=myplot,%
			xtick={1,1.5,2},%
%			xticklabels={-2,-1,..,2},
			ytick={-1,-.5,0},
%			yticklabels = {-2,-1,0,1,2},
%			minor x tick num=1,
%			extra x ticks={.25,.75},%
			ymin=-1.25,ymax=0.1,%
			xmin=.9,xmax=2.1%
]
\addplot [{\colortwo},thick,domain=1:2,samples = 51] {-(x+1)+exp(x-1)};
\addplot [{\colorone}, mark = *] coordinates{ (1,-1) (1.5,-1) (2,-0.75)};
%\legend{True Solution, Euler's Method}
\end{axis}
\node [right] at (myplot.right of origin) {\scriptsize $x$};
\node [above] at (myplot.above origin) {\scriptsize $y$};
\end{tikzpicture}
